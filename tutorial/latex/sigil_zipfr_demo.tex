%% \documentclass[handout,t]{beamer} % HANDOUT
%% \documentclass[handout,notes=show,t]{beamer} % NOTES
\documentclass[t]{beamer} % SLIDES

\usetheme{ZipfR}
\usepackage{beamer-tools}

\input{lib/math}  % basic mathematical notation
\input{lib/stat}  % notation for probability theory and statistics
\input{lib/vector}% convenience macros for vectors and matrices

%%%
%%% local configuration adjustments
%%%

%%% You can change pre-defined colours here, override built-in macros from the
%%% style definition and standard library, as well as define macros needed by
%%% all local documents.

%%% e.g. adjust counterpoint (dark green) for data projectors where greens are
%%% far too bright, as well as green component of light colour and pure green
%%% (of course, it's a better solution to adjust the gamma settings of your monitor)
%%
%% \definecolor{counterpoint}{rgb}{.1, .3, 0}
%% \definecolor{light}{rgb}{.45, .3, .55}
%% \definecolor{puregreen}{rgb}{0, .35, 0}

%% ----- extra packages we need to load

\usepackage{tikz}
\usepackage{alltt}              % code examples with nicely formatted comments


%% ----- automatically show TOC reminder at beginning of each subsection
\AtBeginSubsection[]
{
  \begin{frame}
    \frametitle{Outline}
    \tableofcontents[current,currentsubsection]
  \end{frame}
}

%% ----- some useful macros for R examples

%% > plot(x,y)      \REM{this produces a scatterplot}
\newcommand{\REM}[2][\small]{\textsf{#1\color{primary}\# #2}}

%% nice colour for R output: \begin{Rout} .. \end{Rout}
%% -- ugly hack: I'm sure theres a better way to do this
\newenvironment{Rout}[1][\footnotesize]{%
  \begin{footnotesize}#1\color{dsmblue}\bfseries}{%
  \color{black}\mdseries\end{footnotesize}}

%% ----- other macros -----

%% \IV{condition} ... indicator variable
\newcommand{\IV}[1]{I_{[#1]}}

%% \dpi ... integration over pi
\newcommand{\dpi}[0]{\dX{\pi}} % local adjustments to configuration and macros

%%%%%%%%%%%%%%%%%%%%%%%%%%%%%%%%%%%%%%%%%%%%%%%%%%%%%%%%%%%%%%%%%%%%%%
%% Titlepage

\title[T1: Zipf's Law]{What Every Computational Linguist
  Should Know About Type-Token Distributions and Zipf's Law}
\subtitle{\primary{Tutorial 1, 7 May 2018}}
\author[Stefan Evert]{Stefan Evert\\ FAU Erlangen-Nürnberg}
\date[7 May 2018 | CC-by-sa]{\href{http://zipfr.r-forge.r-project.org/lrec2018.html}{http://zipfr.r-forge.r-project.org/lrec2018.html}\\
 \light{\small Licensed under CC-by-sa version 3.0}}

\begin{document}

\pgfdeclareimage[width=36mm]{lrec-logo}{img/logo_lrec2018}
\pgfdeclareimage[width=30mm]{kallimachos-logo}{img/logo_kallimachos}
\logo{\pgfuseimage{lrec-logo}\hspace{6cm}\pgfuseimage{kallimachos-logo}}

\frame{\titlepage}
\hideLogo{}

%%%%%%%%%%%%%%%%%%%%%%%%%%%%%%%%%%%%%%%%%%%%%%%%%%%%%%%%%%%%%%%%%%%%%%%%
\section{zipfR}

\begin{frame}
  \frametitle{zipfR}
  \framesubtitle{\citet{Evert:Baroni:07}}

  \begin{itemize}
  \item \url{http://zipfR.R-Forge.R-Project.org/}
  \item Conveniently available from CRAN repository
    \begin{itemize}
    \item see Unit 1 for general package installation guides
    \end{itemize}
  \end{itemize}

  \begin{flushright}
    \includegraphics[width=6cm]{img/zipfR_logo}
    \rule{1cm}{0mm}
  \end{flushright}

\end{frame}

\begin{frame}[fragile]
  \frametitle{Loading}

\begin{alltt}
> library(zipfR)

> ?zipfR

> data(package="zipfR")

\REM{package overview in HTML help leads to zipfR tutorial}
> help.start() 
\end{alltt}

\end{frame}


\begin{frame}[fragile]
  \frametitle{Importing data}

\begin{alltt}
> data(ItaRi.spc)     \REM{not necessary in recent package versions}
> data(ItaRi.emp.vgc)

\REM{load your own data sets (see \texttt{?read.spc} etc.\ for file format)}
> my.spc <- read.spc("my.spc.txt")
> my.vgc <- read.vgc("my.vgc.txt")

> my.tfl <- read.tfl("my.tfl.txt")
> my.spc <- tfl2spc(my.tfl) \REM{compute spectrum from frequency list}
\end{alltt}

\end{frame}

\begin{frame}[fragile]
  \frametitle{Looking at spectra}

\begin{alltt}
> summary(ItaRi.spc)
> ItaRi.spc

> N(ItaRi.spc)
> V(ItaRi.spc)
> Vm(ItaRi.spc, 1)
> Vm(ItaRi.spc, 1:5)

\REM{Baayen's P = estimate for slope of VGC}
> Vm(ItaRi.spc, 1) / N(ItaRi.spc)

> plot(ItaRi.spc)
> plot(ItaRi.spc, log="x")
\end{alltt}
\end{frame}

\begin{frame}[fragile]
  \frametitle{Looking at VGCs}

\begin{verbatim}
> summary(ItaRi.emp.vgc)
> ItaRi.emp.vgc

> N(ItaRi.emp.vgc)

> plot(ItaRi.emp.vgc, add.m=1)
\end{verbatim}
\end{frame}

\begin{frame}[fragile]
  \frametitle{Smoothing VGCs with binomial interpolation}
  \framesubtitle{\citep[for details, see][Sec. 2.6.1]{Baayen:01}}

\begin{alltt}
\REM{interpolated VGC}
> ItaRi.bin.vgc <- 
  vgc.interp(ItaRi.spc, N(ItaRi.emp.vgc), m.max=1)

> summary(ItaRi.bin.vgc)

\REM{comparison}
> plot(ItaRi.emp.vgc, ItaRi.bin.vgc,
       legend=c("observed", "interpolated"))
\end{alltt}


\end{frame}


\begin{frame}
  \frametitle{\emph{ultra-}}

  \begin{itemize}
  \item Load the spectrum and empirical VGC of the less common prefix \emph{ultra-}
  \item Compute binomially interpolated VGC for \emph{ultra-}
  \item Plot the binomially interpolated \emph{ri-} and \emph{ultra-}
    VGCs together
  \end{itemize}

\end{frame}


\begin{frame}[fragile]
  \frametitle{Estimating LNRE models}

\begin{alltt}
\REM{fit a fZM model}
\REM{(you can also try ZM and GIGP, and compare them with fZM)}

> ItaUltra.fzm <- lnre("fzm", ItaUltra.spc)

> summary(ItaUltra.fzm)
\end{alltt}

\end{frame}


\begin{frame}[fragile]
  \frametitle{Observed/expected spectra at estimation size}
\begin{alltt}
\REM{expected spectrum}
> ItaUltra.fzm.spc <- 
  lnre.spc(ItaUltra.fzm, N(ItaUltra.fzm))

\REM{compare}
> plot(ItaUltra.spc, ItaUltra.fzm.spc,
       legend=c("observed", "fzm"))

\REM{plot first 10 elements only}
> plot(ItaUltra.spc, ItaUltra.fzm.spc, 
       legend=c("observed", "fzm"), m.max=10)
\end{alltt}
\end{frame}

% \begin{frame}[fragile]
%   \frametitle{Expected spectra at 10 times the estimation size}
% \begin{alltt}
% \REM{extrapolated spectra}

% ItaRi.zm.spc <- lnre.spc(ItaRi.zm, 10*N(ItaRi.zm))

% ItaRi.fzm.spc <- lnre.spc(ItaRi.fzm, 
% 10*N(ItaRi.fzm))

% \REM{compare}

% plot(ItaRi.zm.spc, ItaRi.fzm.spc,
% legend=c("zm","fzm"))
% \end{alltt}
% \end{frame}

% \begin{frame}[fragile]
%   \frametitle{Evaluating extrapolation quality 1}

% \begin{alltt}
% \REM{taking a subsample and estimating a model (if you
% # repat you'll get different sample and different 
% # model!)}

% ItaRi.sub.spc <- sample.spc(ItaRi.spc, N=700000)

% ItaRi.sub.fzm <- lnre("fzm", ItaRi.sub.spc)

% ItaRi.sub.fzm
% \end{alltt}
% \end{frame}


% \begin{frame}[fragile]
%   \frametitle{Evaluating extrapolation quality 2}

% \begin{alltt}
% \REM{extrapolate vgc up to original sample size}

% ItaRi.sub.fzm.vgc <- lnre.vgc(ItaRi.sub.fzm, 
% N(ItaRi.emp.vgc))

% \REM{compare}

% plot(ItaRi.bin.vgc, ItaRi.sub.fzm.vgc,
% N0=N(ItaRi.sub.fzm), legend=c("interpolated","fZM"))
% \end{alltt}
% \end{frame}


% \begin{frame}[fragile]
%   \frametitle{Compare growth of two categories 1}
% \begin{alltt}
% \REM{the ultra- prefix}

% data(ItaUltra.spc)

% summary(ItaUltra.spc)

% \REM{cf.}

% summary(ItaRi.spc)

% \REM{estimating model}

% ItaUltra.fzm <- lnre("fzm",ItaUltra.spc,exact=F)

% ItaUltra.fzm

% \end{alltt}
% \end{frame}


\begin{frame}[fragile]
  \frametitle{Compare growth of two categories}
\begin{alltt}
\REM{extrapolation of \emph{ultra-} VGC to sample size of \emph{ri-} data}
> ItaUltra.ext.vgc <- 
  lnre.vgc(ItaUltra.fzm, N(ItaRi.emp.vgc))

\REM{compare}
> plot(ItaUltra.ext.vgc, ItaRi.bin.vgc,
       N0=N(ItaUltra.fzm), legend=c("ultra-", "ri-"))

\REM{zooming in}
> plot(ItaUltra.ext.vgc, ItaRi.bin.vgc,
       N0=N(ItaUltra.fzm), legend=c("ultra-", "ri-"),
       xlim=c(0, 100e3))
\end{alltt}
\end{frame}


\begin{frame}[fragile]
  \frametitle{Model validation by parametric bootstrapping}
\begin{alltt}
\REM{define function to extract relevant information from fitted model}
> extract.stats <- function (m)
    data.frame(alpha=m$param$alpha, A=m$param$A, 
               B=m$param$B, S=m$S, X2=m$gof$X2)

\REM{run bootstrapping procedure (default = 100 replicates)}
> runs <- lnre.bootstrap(ItaUltra.fzm, N(ItaUltra.fzm), 
                         lnre, extract.stats, type="fzm")

> head(runs)

\REM{NB: don't try this with large samples (N \(>\) 1M tokens)}
\end{alltt} %$
\end{frame}

\begin{frame}[fragile]
  \frametitle{Model validation by parametric bootstrapping}
\begin{alltt}
\REM{distribution of estimated model parameters}
> hist(runs$alpha, freq=FALSE, xlim=c(0, 1))
> lines(density(runs$alpha), lwd=2, col="red")
> abline(v=ItaUltra.fzm$param$alpha, lwd=2, col="blue")

\REM{try the other parameters for yourself!}

\REM{distribution of goodness-of-fit values}
> hist(runs$X2, freq=FALSE)
> lines(density(runs$X2), lwd=2, col="red")

\REM{estimated population vocabulary size}
> hist(runs$S) \REM{what is wrong here?}
\end{alltt} %$
\end{frame}


%%%%%%%%%%%%%%%%%%%%%%%%%%%%%%%%%%%%%%%%%%%%%%%%%%%%%%%%%%%%%%%%%%%%%%
%% References (if any)

\frame[allowframebreaks]{
  \frametitle{References}
  \bibliographystyle{natbib-stefan}
  \begin{scriptsize}
    \bibliography{stefan-publications,stefan-literature}
  \end{scriptsize}
}

\end{document}
